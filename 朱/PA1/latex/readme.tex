\documentclass{ctexart}
\usepackage{amsmath, mathrsfs, amsfonts}
\usepackage{tikz}
\usepackage{graphicx}
\usepackage{xcolor}
\usepackage{subcaption}
\usepackage{adjustbox}
\usepackage{titlesec}
\usepackage{geometry}
\usepackage{listings}
\geometry{left=2cm, right=2cm, top=2cm, bottom=2cm}
% library
\usetikzlibrary{trees,mindmap,positioning}
% 修改图标题的格式
\captionsetup[figure]{name=Fig.,labelsep=period}
% 自定义section标题格式
\titleformat{\section}
  {\normalfont\large\bfseries} % 节标题的字体样式
  {\thesection} % 节编号
  {2em} % 节编号和标题文本之间的距离
  {} % 节标题前的代码,留空表示无额外代码

\lstset{
    language=C++, % 设置语言为cpp
    basicstyle=\ttfamily, % 使用等宽字体
    keywordstyle=\color{red}, % 关键词颜色
    commentstyle=\color{green}, % 注释颜色
    stringstyle=\color{violet}, % 字符串颜色
    breaklines=true, % 自动换行
    numbers=left, % 行号在左侧显示
    numberstyle=\tiny\color{gray}, % 行号样式
    %frame=single, % 代码框线样式
    showstringspaces=false 
}

\begin{document}

\begin{center}
    \begin{tabular}{ccc}
    \Large{编译原理PA1} \\
    20238132022 物联网2班 朱彦渊 \\
    \today
    
    \end{tabular}
    \end{center}
所选数据结构: 队列\\

所有代码均保存在queue.cl中。由于在linux console下使用中文输入可能会导致乱码(本来写了,但是写完注释后面居然自动乱了,直接又跑不了了TAT,后面改了好久才回来),所以不在代码中写注释,对部分代码的解释转到文档中进行。
代码分为三个部分,分别是Node类 Queue类 和Main类。其中Node作为Queue的节点,有Data和Next两个变量。Queue为队列本身,有head,tail,size三个变量,以及课件要求的功能函数.
cool语言的类中的变量应该默认为private的,所以在定义Node类的时候我使用了setNext(),  getData(), getNext()三个函数进行辅助。由于偷懒,队列的元素默认为Int型,如果使用Object型的话,应该要在输出的时候分类调用不同的IO方法。
Main函数用于测试队列,该程序的运行结果如下图所示,符合课件要求的最小功能集.
\begin{figure}[htp]
    \centering
    \includegraphics[width=0.5\textwidth]{res.png}
    \caption{程序运行结果图}
\end{figure}

\indent 完整代码如下:\\

\begin{lstlisting}
    -- 队列节点
class Node inherits Object{
  data : Int;
  next : Node;

  init(d : Int, n : Node) : Node {
    {
      data <- d;
      next <- n;
      self;
    }
  };

  setNext( n : Node) : Object{
    {
      next <- n;
      self;
    }
  };

  getData() : Int{
    data
  };

  getNext() : Node{
    next
  };

};

--队列
class Queue inherits IO{
  head : Node;
  tail : Node;
  size : Int;
  init() :  Queue{
    {
      size <- 0;
      self;
    }
  };

  isEmpty() : Bool {
    isvoid head
  };

  enqueue (data : Int) : Object {
    {
    let n : Node in 
    let newNode : Node <- (new Node).init(data, n) in
    {
      if isEmpty() then{
        head <- newNode;
        tail <- newNode;
      }
      else{
        tail.setNext(newNode);
        tail <- newNode;
      }fi;

      size <- size + 1;
    };
    }
  };

  dequeue () : Int{
    if isEmpty() then
      {
        out_string("Error! Can't dequeue an empty queue!\n");
        0;
      }
    else{
      let data2rm : Int <- head.getData() in {
        head <- head.getNext();
        size <- size - 1;

        data2rm;
      };
    }
    fi
  };


  front () : Int{
    if isEmpty() then
      {
        out_string("Error! Empty queue!\n");
        0;
      }
    else{
      head.getData();
    }
    fi
  };


  print () : Object {
    if isEmpty() then{
      out_string("empty queue!\n");
    }
    else{
      let curr : Node <- head in{
        out_string("the elemnets in the queue:\n");
        while not (isvoid curr) loop{
          out_int(curr.getData());
          out_string(" ");
          curr <- curr.getNext();
        }pool;
        out_string("\n");
        out_string("size of the queue: ");
        out_int(size);
        out_string("\n");
      };
    } fi
    
  };
};

-- Main函数,用于测试.
class Main inherits IO {
  main() : Object {
    let q : Queue <- new Queue in {

      if q.isEmpty() then 
        out_string("empty queue.\n")
      else
        out_string("not empty.\n")
      fi;

      out_string("enqueued : 1,2,3\n");
      q.enqueue(1);
      q.enqueue(2);
      q.enqueue(3);
      q.print();

      out_string("the front of the queue is : ");
      out_int(q.front());
      out_string("\n");
      q.print();

      q.dequeue();
      out_string("after dequeue: ");
      q.print();


    }
  };
};


\end{lstlisting}
\end{document}